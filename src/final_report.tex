\documentclass{article}

% if you need to pass options to natbib, use, e.g.:
%     \PassOptionsToPackage{numbers, compress}{natbib}
% before loading neurips_2024


% ready for submission
\usepackage[preprint]{neurips_2024}
\usepackage{makecell}
\usepackage{enumitem}
\usepackage{subcaption}

% to compile a preprint version, e.g., for submission to arXiv, add add the
% [preprint] option:
%     \usepackage[preprint]{neurips_2024}


% to compile a camera-ready version, add the [final] option, e.g.:
%     \usepackage[final]{neurips_2024}


% to avoid loading the natbib package, add option nonatbib:
%    \usepackage[nonatbib]{neurips_2024}


\usepackage[utf8]{inputenc} % allow utf-8 input
\usepackage[T1]{fontenc}    % use 8-bit T1 fonts
\usepackage{hyperref}       % hyperlinks
\usepackage{url}            % simple URL typesetting
\usepackage{booktabs}       % professional-quality tables
\usepackage{amsfonts}       % blackboard math symbols
\usepackage{nicefrac}       % compact symbols for 1/2, etc.
\usepackage{microtype}      % microtypography
\usepackage{xcolor}         % colors
\usepackage{graphicx}       % for showing images
\usepackage{subcaption}
\usepackage{float}
\usepackage{placeins}

\title{Midway Report -- Exploration of Migration Patterns and Prediction of Sightings for Osprey, California Condor, and Puffin Species}


% The \author macro works with any number of authors. There are two commands
% used to separate the names and addresses of multiple authors: \And and \AND.
%
% Using \And between authors leaves it to LaTeX to determine where to break the
% lines. Using \AND forces a line break at that point. So, if LaTeX puts 3 of 4
% authors names on the first line, and the last on the second line, try using
% \AND instead of \And before the third author name.


\author{%
    Iris Glaze \\
    \texttt{kiglaze@ncsu.edu} \\
}



\begin{document}

    \maketitle

    % Added sections for project report
    \section{Background \& Introduction}
    The Cornell Lab of Ornithology maintains a large database of bird observations called EBird.
    Birdwatchers submit their bird observations through the Cornell Lab of
    Ornithology mobile app or website, so the data is crowd-sourced. Data points
    exist for thousands of bird species. [7]

    This study is focused on Ospreys, California Condors, and Atlantic Puffins in
    the United States, Mexico, Canada, and Cuba. Ospreys and Atlantic Puffins both
    migrate, but California Condors do not. California Condors are endangered, and Atlantic
    Puffins have a vulnerable population but are not endangered. Ospreys currently
    have a low concern level regarding their population, but their population has previously
    been under threat in the 1950’s. Given this variety of bird characteristics,
    models can be created to predict the presence of bird sightings during a given
    week and location for each of these bird species.

    Limitations of studying the eBird data include that observations depend on
    humans being present and able to access Cornell’s bird app or website. More
    humans birdwatching may be associated with more observations in the dataset.
    The data also depends on people using the bird app to use the app properly.

    Predicting the presence of bird species sightings per week is useful to better understand the
    ecological makeup of a geographical region in relation to time. It also
    provides information about the bird species’ migration patterns in relation to
    a particular geographic region, and the health of the bird population.

    Martin’s paper uses random forests as one of the machine learning models to predict special-temporal
    patters. [1] This helps inform one of the models that this research will use.

    \section{Methods}
Osprey, Atlantic Puffin, and California observation data was first studied within a
larger geographical region (United States, Mexico, Canada, and Cuba). Then, the
geographical area of focus was narrowed and restricted for each bird species
for the next phase of the study.

To study the Osprey, Atlantic Puffin, and California observation data at a larger geographic
scale, corresponding .csv file geographical observation data was grouped by
observation date week and year combinations for all observation dates between
March 1, 2023 and August 31, 2025. Then it was mapped onto maps of North
America by latitude and longitude. Observation countries of focus were limited
to the United States, Canada, Mexico, and Cuba. A similar process was followed
for this data to also demonstrate maps showing bird sightings grouped by season
and year for each listed species. Because the number of observations per annual
season for some bird species was too large to visualize on a map, samples of
the total number of seasonal observations were taken for the purposes of data
visualization. Here are the seasonal sample percentages shown by species below:

    \begin{table}[h]
        \centering
        \begin{tabular}{l c}
            \toprule
            \textbf{Species} & \textbf{Percentage Sampled (Per Annual Season)} \\
            \midrule
            Osprey & 10\% \\
            Atlantic Puffin & 40\% \\
            California Condor & 100\% \\
            \bottomrule
        \end{tabular}
    \end{table}

These geographical distribution maps were stitched together in chronological order to
make animations showing seasonal migration patterns. This resulted in the creation
of the following types of map visualization animations using raw data:
    \begin{itemize}[noitemsep, topsep=0pt]
        \item Osprey weekly sighting maps
        \item Osprey seasonal sighting maps (sampled at 10\%)
        \item Atlantic Puffin weekly sighting maps
        \item Atlantic Puffin seasonal sighting maps (sampled at 40\%)
        \item California Condor weekly sighting maps
        \item California Condor seasonal sighting maps (100\% of seasonal sighting data points)
    \end{itemize}

To clean up the geographical observation data map visualizations and thus the
corresponding map animations, all six of the previously listed animations were
re-created, but with “noise” points being excluded by applying DBSCAN. The
distance calculation used for DBSCAN was the “Haversine” distance, which
accounts for the nature of latitude, longitude, and the roundness of the Earth.
DBSCAN was applied for the data points in each individual map making up an
animation. These post-DBSCAN geographical animations provided a clearer
visualization of bird migration pattern clusters by species.

Next, geographic regions of focus were narrowed for the remainder of the study. A
specific latitude and longitude range was singled out for each of the three
species, Osprey, Atlantic Puffins, and California Condors, using the generated
visualization maps and animations. The following geographical location and bird
species combinations were human-selected for further examination: Ospreys in
the Eastern Glacier Bay area in Alaska, Atlantic Puffins in the Massachusetts area,
and California Condors in the Grand Canyon area of Arizona. These hand-selected
geographic regions of focus are more clearly defined by latitude and longitude
ranges, which are demonstrated in the figures below.

    \begin{table}[h]
        \centering
        \begin{tabular}{l l c c c c}
            \toprule
            \textbf{Bird Species} &
            \makecell{\textbf{Region}\\\textbf{Label}} &
            \makecell{\textbf{Minimum}\\\textbf{Latitude}} &
            \makecell{\textbf{Maximum}\\\textbf{Latitude}} &
            \makecell{\textbf{Minimum}\\\textbf{Longitude}} &
            \makecell{\textbf{Maximum}\\\textbf{Longitude}} \\
            \midrule

            \textbf{Osprey} &
            \makecell[l]{Eastern\\Glacier Bay\\Area, Alaska} &
            58.0 & 60.0 & -137.0 & -135.0 \\[2pt]

            \textbf{Atlantic Puffin} &
            \makecell[l]{Massachusetts\\Area} &
            41.0 & 42.8 & -73.7 & -69.8 \\[2pt]

            \textbf{California Condor} &
            \makecell[l]{Grand Canyon\\Area, Arizona} &
            35.0 & 37.0 & -113.0 & -111.0 \\
            \bottomrule
        \end{tabular}
    \end{table}

These map visualizations show the selected geographic regions of focus more specifically:

    \begin{figure}[H]
        \centering
        % First row
        \begin{subfigure}{0.48\textwidth}
            \includegraphics[width=\textwidth]{images/geographic_regions/osprey_region.png}
            \caption{
                Osprey Region of Focus - Eastern Glacier Bay Area, Alaska.\\
                Latitude Range: (58.0, 60.0).\\
                Longitude Range: (-137.0, -135.0).
            }
            \label{fig:geo_reg_osprey}
        \end{subfigure}
        \hfill
        \begin{subfigure}{0.48\textwidth}
            \includegraphics[width=\textwidth]{images/geographic_regions/atl_puffin_region.png}
            \caption{
                Atlantic Puffin Region of Focus - Massachusetts Area.\\
                Latitude Range: (41.0, 42.8).\\
                Longitude Range: (-73.7, -69.8).
            }
            \label{fig:geo_reg_atl_puffin}
        \end{subfigure}

        \vspace{0.5em}

        % Second row
        \begin{subfigure}{0.48\textwidth}
            \includegraphics[width=\textwidth]{images/geographic_regions/ca_condor_region.png}
            \caption{
                California Condor Region of Focus - Grand Canyon Area, Arizona.\\
                Latitude Range: (35.0, 37.0).\\
                Longitude Range: (-113.0, -111.0).
            }
            \label{fig:geo_reg_ca_condor}
        \end{subfigure}

        \caption{Geographical Regions Selected - Per Species}
        \label{fig:geo_reg_grid}
    \end{figure}

For each of these bird species and geographical area combinations, data was transformed to
determine whether bird species sightings existed in the corresponding
geographical areas each week between March 2015 and August 2025. This was
created to help predict whether a bird species is spotted in its geographical
region of focus given a week. So, the \textit{existence} of a bird sighting for a
given week and region was studied rather than the \textit{number} of sightings during
a week. This design choice was made because some sightings in the same day
could be submitted by different people but refer to the same bird on the same
day.

To do this, first this data was fit to a cyclic logistic regression model for each species and
geographic region combination to predict likelihoods based on the week number
alone. Next, tree-based classification models were used to predict if a bird was
be spotted during a week, given its environmental and weather data. These
tree-based classification models included Decision Trees with pruning, Random
Forest, Gradient Boosting Classifier, and Extreme Gradient Boosting (XGBoost).

Weekly data was used for the tree classifiers instead of daily or seasonal data, because it
seemed like a more reasonable increment of time to study. For instance, bird
sighting presence or absence data had a nice mix of sighted vs. not sighted
values for both Osprey and Atlantic Puffins when the observation dates were grouped
by week instead of by day. Also, the migration map animations that had one map
per week showed a nice speed of the migration pattern that was subjectively
neither too fast nor too slow.

Features of the data that these tree models examined as input included: a continuous numeric
representation of seasonality (labelled as week\_cos), average temperature (C),
minimum temperature (C), maximum temperature (C), precipitation (mm), wind
speed (km/hour), air pressure (hPa), and snow (mm).

The dependent variable is whether a bird was observed in its geographic region of focus, and
each row of data represented a week and year combination.

\section{Plan and Experiment Setup}

A large 200GB dataset of bird sighting data, named Basic Dataset, was downloaded
from \href{https://ebird.org/data/download}{https://ebird.org/data/download} after requesting
access permission from Cornell’s Lab of Ornithology. This data includes
latitude, longitude, bird species name, observation date, country name, some
additional variables describing the geographical location, and sometimes the
bird’s behavior. Next, a Python script was used to extract all data for
specific species of birds by filtering the large dataset in chunks. This data
was exported it to a .csv file. The set of bird species included were Ospreys,
California Condors, and Atlantic Puffins. Then, the Osprey observations,
California Condor observations, and Atlantic Puffin observations were each
output to their own single .csv file.

After bird species were sorted into their own .csv files, additional columns were
added to describe the observation dates to make for easier grouping by dates.
Some of these included the month, the year, the season, and the week number.
Columns for the latitude and longitude in radians were also added to help with
DBSCAN calculations.

A dataset was created for each bird species which contained each week and year combination
between March 1, 2015 and August 31, 2025. Another Boolean column was added to
indicate if the bird was spotted in its geographic region of focus during that
given week or not.

Next, environmental and weather data obtained from the MeteoStat Python package was joined with this
dataset which indicated if a bird was spotted or not. This provided more
variables for additional model exploration of each studied species in their
respective geographical location. These additional variables helped predict the
presence or absence of bird species sightings in their chosen geographic region
per week in a more detailed fashion.

This data setup was created to assist with revealing information about the migration patterns
of Osprey, Atlantic Puffins, and California Condors. It also explores information to help bird watchers know when spot:

\begin{itemize}
    \item Osprey in the Eastern Glacier Bay area of Alaska
    \item Atlantic Puffins in the general Massachusetts, Connecticut, or Rhode Island area,
    including associated coastal regions
    \item California Condors in a selected Grand Canyon region in Northern Arizona
\end{itemize}

We hypothesized that Osprey would be visible in the corresponding Glacier Bay region in the summertime,
Atlantic Puffins would be most visible in their chosen Massachusetts region in
Spring and Fall, and California Condors would be occasionally spotted
throughout the year in the Grand Canyon area, regardless of the time of year.

    \section{Results}
    These example bird observation maps show DBSCAN being applied. The first example shown in Figure~\ref{fig:osprey_weekly_maps} is from observation days
    being grouped by week, and the second example shown in Figure~\ref{fig:osprey_seasonal_maps} has days grouped by season.\\

    \begin{figure}[htbp]
        \centering
        \textbf{Week 33, 2024: Osprey Observation Maps}\\[0.5em]
        \begin{subfigure}{0.48\textwidth}
            \centering
            \includegraphics[width=\textwidth]{images/no_dbscan_osprey_week33_2024.png}
            \caption{Without DBSCAN}
        \end{subfigure}\hfill
        \begin{subfigure}{0.48\textwidth}
            \centering
            \includegraphics[width=\textwidth]{images/dbscan_osprey_week33_2024.png}
            \caption{With DBSCAN}
        \end{subfigure}
        \caption{Osprey observation maps for week 33, 2024, with and without DBSCAN.}
        \label{fig:osprey_weekly_maps}
    \end{figure}
    \begin{figure}[htbp]
        \centering
        \textbf{Winter 2023: Osprey Observation Maps}\\[0.5em]
        \begin{subfigure}{0.48\textwidth}
            \centering
            \includegraphics[width=\textwidth]{images/no_dbscan_osprey_winter_2023.png}
            \caption{Without DBSCAN}
        \end{subfigure}\hfill
        \begin{subfigure}{0.48\textwidth}
            \centering
            \includegraphics[width=\textwidth]{images/dbsan_osprey_winter_2023.png}
            \caption{With DBSCAN}
        \end{subfigure}
        \caption{Osprey observation maps for week 33, 2024, with and without DBSCAN.}
        \label{fig:osprey_seasonal_maps}
    \end{figure}

    Animations linked using YouTube below in Table~\ref{tab:post-dbscan-map-animations} demonstrate the DBSCAN maps over time from March 2023-August
    2025 for each species. The animations helped inform selecting locations of
    further focus for the next step, cyclic logistic regression.\\

    \begin{table}[H]
        \centering
        \caption{Post-DBSCAN Map Animations}
        \label{tab:post-dbscan-map-animations}

        \begin{tabular}{l}
            \textbf{Weekly observation date groupings:} \\
            Osprey, DBSCAN, Weekly: \url{https://youtu.be/dAQBNUsSv2A} \\
            Atlantic Puffin, DBSCAN, Weekly: \url{https://youtu.be/eX5zdkZRiiQ} \\
            California Condor, DBSCAN, Weekly: \url{https://youtu.be/-5MAU0RcSOI} \\
            \\[-0.3em]
            \textbf{Seasonal observation date groupings:} \\
            Osprey, DBSCAN, Seasonal: \url{https://youtu.be/8nmQlWG9Sa4} \\
            Atlantic Puffin, DBSCAN, Seasonal: \url{https://youtu.be/Ih1CqyifmHI} \\
            California Condor, DBSCAN, Seasonal: \url{https://youtu.be/GDQsCA-IGp8} \\
        \end{tabular}

    \end{table}

    Cyclical logistic regression was carried out for the
    presence of Osprey, Atlantic Puffin, and California Condor sightings given the
    week number. These plots are shown in Figure~\ref{fig:cyc_log_reg_grid}. The Osprey plot demonstrates a peak in the likelihood of viewing
    at least one Osprey in Glacier Bay around approximately week 29 or week 30,
    which is mid to late July. The Atlantic Puffin plot peaks in sighting
    likelihood around the end and beginning of the calendar year. The California
    Condor plot doesn’t have as much variation in the likelihood of sightings throughout
    the year, but California Condors are likely to be spotted each week in the
    Grand Canyon region observed.

    \begin{figure}[H]
        \centering
        % First row
        \begin{subfigure}{0.48\textwidth}
            \includegraphics[width=\textwidth]{images/log_reg_osprey_glacier_bay,_alaska.png}
            \caption{Osprey: Glacier Bay, Alaska}
            \label{fig:cyc_log_reg_osprey}
        \end{subfigure}
        \hfill
        \begin{subfigure}{0.48\textwidth}
            \includegraphics[width=\textwidth]{images/log_reg_atlantic_puffin_massachusetts_coastal_area.png}
            \caption{Atlantic Puffin: Massachusetts Coastal Area}
            \label{fig:cyc_log_reg_puffin}
        \end{subfigure}

        \vspace{0.5em}

        % Second row
        \begin{subfigure}{0.48\textwidth}
            \includegraphics[width=\textwidth]{images/log_reg_california_condor_grand_canyon.png}
            \caption{California Condor: Grand Canyon}
            \label{fig:cyc_log_reg_condor}
        \end{subfigure}

        \caption{Cyclical Logistic Regression: Likelihood of Bird Observation Occurrence in Specified Geographical Regions by Week for Three Species/Locations}
        \label{fig:cyc_log_reg_grid}
    \end{figure}

Decision Trees with pruning showed some information about how feature values could help
predict the presence of absence of the various bird species. The diagrams showing these decision trees are shown in Figure~\ref{fig:dec_tree_grid}.
Osprey were more likely to be spotted in the Eastern Glacier Bay region of focus during a week
when the average temperature was over 4.16 degrees Celsius, and less likely to
be spotted when the week is NOT in the middle of the Summer (week\_cos >
-0.733). Atlantic Puffins were more likely to be spotted in the specified
Massachusetts area during the middle of Winter (week\_cos > 0.857), or if
there is more than 5.941 mm of precipitation during a given week. California
Condors tended to be more likely to be spotted in the Grand Canyon region of
focus when it’s NOT in the middle of Winter (week\_cos <= .474) or when the
maximum temperature exceeds 13.15 Celsius AND the wind speed exceeds 8.683 km/hour.

The Random Forest model was run for each species and geographical region
combination, and it indicated the most important deciding features for predicting
the presence vs. absence of a bird. Graphical representations of these results are shown in Figure~\ref{fig:rand_forest_grid}. The most important deciding feature for the
presence or absence of Osprey was average temperature, followed by minimum
temperature. The seasonality continuous variable week\_cos placed third. For
Atlantic Puffins, the two most important deciding features were seasonality
(week\_cos) and then the amount of precipitation. The most important features
for predicting presence vs. absence of California Condors included: maximum
temperature, average temperature, and seasonality (week\_cos).

The predicting power of all four tree classification models was compared for all
three bird species, and these metrics are shown in Table~\ref{tab:all_species_metrics}.
For Osprey, Decision Tree and Gradient Boosting are best models overall, for
predicting “Present” observation. Decision Tree and XGBoost are best for
predicting “Absent” observation. In the case of Atlantic Puffins, the Decision
Tree with pruning and Random Forest are the best models overall for predicting
their presence. Decision Tree, Random Forest, and XGBoost are the best for
predicting an absence of Atlantic Puffin sightings. For California Condors, a
Decision Tree and XGBoost are the best models for predicting the presence of
California Condor sightings. Also, for California Condors, the Decision Tree
and Random Forest models are the best for predicting an “Absent” observation.

% tighten tables
    \renewcommand{\arraystretch}{0.85}
    \setlength{\tabcolsep}{3pt}

    \renewcommand{\arraystretch}{0.85}
    \setlength{\tabcolsep}{3pt}

    \begin{table}[h!]
        \centering
        \footnotesize

        % ----- Subtable A -----
        \begin{subtable}{0.48\textwidth}
            \centering
            \begin{tabular}{lcccc}
                \toprule
                & \textbf{Dec. Tree} & \textbf{RF} & \textbf{GB} & \textbf{XGB} \\
                \midrule
                Accuracy & 0.652 & 0.667 & 0.776 & \textbf{0.782} \\
                True–Recall & \textbf{0.850} & 0.750 & 0.519 & 0.26 \\
                True–Precision & 0.274 & 0.268 & \textbf{0.368} & 0.30 \\
                True–F1 & 0.415 & 0.395 & \textbf{0.431} & 0.28 \\
                False–Recall & 0.619 & 0.653 & 0.826 & \textbf{0.88} \\
                False–Precision & \textbf{0.961} & 0.939 & 0.898 & 0.86 \\
                False–F1 & 0.753 & 0.770 & 0.860 & \textbf{0.87} \\
                Macro F1 & 0.584 & 0.582 & \textbf{0.646} & 0.58 \\
                Balanced Acc. & \textbf{0.734} & 0.701 & 0.672 & 0.572 \\
                \bottomrule
            \end{tabular}
            \caption{Osprey Models}
            \label{tab:osprey_metrics}
        \end{subtable}
        \hfill
        % ----- Subtable B -----
        \begin{subtable}{0.48\textwidth}
            \centering
            \begin{tabular}{lcccc}
                \toprule
                & \textbf{Dec. Tree} & \textbf{RF} & \textbf{GB} & \textbf{XGB} \\
                \midrule
                Accuracy & 0.768 & \textbf{0.783} & 0.685 & 0.703 \\
                True–Recall & \textbf{0.600} & 0.533 & 0.372 & 0.23 \\
                True–Precision & 0.474 & \textbf{0.500} & 0.390 & 0.38 \\
                True–F1 & \textbf{0.529} & 0.516 & 0.381 & 0.29 \\
                False–Recall & 0.815 & 0.852 & 0.795 & \textbf{0.87} \\
                False–Precision & \textbf{0.880} & 0.868 & 0.782 & 0.76 \\
                False–F1 & 0.846 & \textbf{0.860} & 0.789 & 0.81 \\
                Macro F1 & \textbf{0.688} & \textbf{0.688} & 0.585 & 0.55 \\
                Balanced Acc. & \textbf{0.707} & 0.693 & 0.584 & 0.551 \\
                \bottomrule
            \end{tabular}
            \caption{Atlantic Puffin Models}
            \label{tab:puffin_metrics}
        \end{subtable}

        \vspace{8pt}

        % ----- Subtable C -----
        \begin{subtable}{0.7\textwidth}
            \centering
            \begin{tabular}{lcccc}
                \toprule
                & \textbf{Dec. Tree} & \textbf{RF} & \textbf{GB} & \textbf{XGB} \\
                \midrule
                Accuracy & 0.891 & 0.920 & 0.939 & \textbf{0.946} \\
                True–Recall & 0.896 & 0.933 & 0.963 & \textbf{0.97} \\
                True–Precision & \textbf{0.992} & 0.984 & 0.975 & 0.97 \\
                True–F1 & 0.941 & 0.958 & 0.969 & \textbf{0.97} \\
                False–Recall & \textbf{0.750} & 0.500 & 0.200 & 0.20 \\
                False–Precision & 0.176 & \textbf{0.182} & 0.143 & 0.17 \\
                False–F1 & \textbf{0.286} & 0.267 & 0.167 & 0.18 \\
                Macro F1 & \textbf{0.613} & 0.612 & 0.568 & 0.58 \\
                Balanced Acc. & \textbf{0.823} & 0.716 & 0.581 & 0.584 \\
                \bottomrule
            \end{tabular}
            \caption{California Condor Models}
            \label{tab:condor_metrics}
        \end{subtable}

        \caption{Accuracy, Recall, Precision, F1-score, and Balanced Accuracy for all three bird species. Models compared include: Decision Tree with pruning, Random Forest, Gradient Boosting Classifier, and Extreme Gradient Boosting (XGBoost).}
        \label{tab:all_species_metrics}
    \end{table}


    \FloatBarrier
    \section{Conclusion}
Although California Condors had many fewer sightings than Osprey overall, they were
consistently likely to be spotted in the Grand Canyon at least once a week, any
given week during the year. Ospreys are much more abundant overall than
California Condors, as seen in the migration map animations, but only likely to
be spotted in the Eastern Glacier Bay region in Alaska during the warmer
months. Similarly, the DBSCAN maps animations which show bird observations and
migration patterns support what is shown in the cyclic logistic regression
plots in both cases.

However, the Atlantic Puffin model results revealed the most surprising results. Sightings
of Atlantic Puffins in the Massachusetts coastal area were more common during
the winter weeks. This aligns partially with how the DBSCAN map animations of
Atlantic Puffin migrations in the sense that Atlantic Puffins tend to migrate
further North than Massachusetts in the summer, in Maine and coastal
South-Eastern Canada. In addition to the winter weeks, Atlantic Puffins were
also more likely to be spotted during non-winter weeks which had more than 5.941
mm of precipitation.

    \section{GitHub Link}
    \href{https://github.com/kiglaze_ncstate/ncsu-engr-ALDA-F25-Project-P14}{\url{https://github.com/kiglaze_ncstate/ncsu-engr-ALDA-F25-Project-P14}}

    \section*{References}
    \medskip
    {
        \small
        [1] Martín, B., J. González-Arias, and J. A. Vicente-Vírseda.
    "Machine learning as a successful approach for predicting complex
    spatio-temporal patterns in animal species abundance." Machine learning 44
        (2021): 289-301.

        [2] Fuentes, Miguel, et al. "BirdFlow: Learning seasonal
    bird movements from eBird data." Methods in Ecology and Evolution 14.3
        (2023): 923-938.

        [3] Ahmed, dis- Nahian, et al. "Spatial clustering of
    citizen science data improves downstream species tribution models."
    Proceedings of the AAAI Conference on Artificial Intelligence. Vol. 39. No. 27.
    2025.

        [4] Cornell Lab. Osprey. All About Birds. \url{https://www.allaboutbirds.org/guide/Osprey/}

    [5] Cornell Lab. California Condor. All About Birds. \url{https://www.allaboutbirds.org/guide/California_Condor/}

    [6] Cornell Lab. Atlantic Puffin. All About Birds. \url{https://www.allaboutbirds.org/guide/Atlantic_Puffin/}

    [7] eBird. About eBird. \url{https://ebird.org/about}
    }

    \appendix
    \section{Appendix}
    \begin{figure}[H]
        \centering
        % First row
        \begin{subfigure}{0.60\textwidth}
            \centering
            \includegraphics[width=1.0\textwidth]{images/decision_trees/osprey_dec_tree.png}
            \caption{Decision tree for Osprey weekly sightings.}
            \label{fig:osprey_dec_tree2}
        \end{subfigure}
        \hfill
        \begin{subfigure}{0.36\textwidth}
            \centering
            \includegraphics[width=1.2\textwidth]{images/decision_trees/atl_puffin_dec_tree.png}
            \caption{Decision tree for Atlantic Puffin weekly sightings.}
            \label{fig:atl_puff_dec_tree2}
        \end{subfigure}

        \vspace{0.5em}

        % Second row
        \begin{subfigure}{0.90\textwidth}
            \centering
            \includegraphics[width=1.0\textwidth]{images/decision_trees/ca_condor_dec_tree.png}
            \caption{Decision tree for California Condor weekly sightings.}
            \label{fig:ca_condor_dec_tree2}
        \end{subfigure}

        \caption{Decision Trees With Pruning}
        \label{fig:dec_tree_grid}
    \end{figure}

    \begin{figure}[H]
        \centering
        % First row
        \begin{subfigure}{0.48\textwidth}
            \includegraphics[width=\textwidth]{images/random_forest_feature_rankings/random_forest_feature_importances_osprey.png}
            \caption{Osprey: Glacier Bay, Alaska}
            \label{fig:rand_forest_osprey}
        \end{subfigure}
        \hfill
        \begin{subfigure}{0.48\textwidth}
            \includegraphics[width=\textwidth]{images/random_forest_feature_rankings/random_forest_feature_importances_atlantic_puffin.png}
            \caption{Atlantic Puffin: Massachusetts Coastal Area}
            \label{fig:rand_forest_puffin}
        \end{subfigure}

        \vspace{0.5em}

        % Second row
        \begin{subfigure}{0.48\textwidth}
            \includegraphics[width=\textwidth]{images/random_forest_feature_rankings/random_forest_feature_importances_california_condor.png}
            \caption{California Condor: Grand Canyon}
            \label{fig:rand_forest_condor}
        \end{subfigure}

        \caption{Random Forest: Most Important Features}
        \label{fig:rand_forest_grid}
    \end{figure}

\end{document}
