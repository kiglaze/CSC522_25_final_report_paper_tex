\documentclass{article}

% if you need to pass options to natbib, use, e.g.:
%     \PassOptionsToPackage{numbers, compress}{natbib}
% before loading neurips_2024


% ready for submission
\usepackage[preprint]{neurips_2024}


% to compile a preprint version, e.g., for submission to arXiv, add add the
% [preprint] option:
%     \usepackage[preprint]{neurips_2024}


% to compile a camera-ready version, add the [final] option, e.g.:
%     \usepackage[final]{neurips_2024}


% to avoid loading the natbib package, add option nonatbib:
%    \usepackage[nonatbib]{neurips_2024}


\usepackage[utf8]{inputenc} % allow utf-8 input
\usepackage[T1]{fontenc}    % use 8-bit T1 fonts
\usepackage{hyperref}       % hyperlinks
\usepackage{url}            % simple URL typesetting
\usepackage{booktabs}       % professional-quality tables
\usepackage{amsfonts}       % blackboard math symbols
\usepackage{nicefrac}       % compact symbols for 1/2, etc.
\usepackage{microtype}      % microtypography
\usepackage{xcolor}         % colors
\usepackage{graphicx}       % for showing images
\usepackage{subcaption}
\usepackage{float}


\title{Midway Report -- Exploration of Migration Patterns and Prediction of Sightings for Osprey, California Condor, and Puffin Species}


% The \author macro works with any number of authors. There are two commands
% used to separate the names and addresses of multiple authors: \And and \AND.
%
% Using \And between authors leaves it to LaTeX to determine where to break the
% lines. Using \AND forces a line break at that point. So, if LaTeX puts 3 of 4
% authors names on the first line, and the last on the second line, try using
% \AND instead of \And before the third author name.


\author{%
    Iris Glaze \\
    \texttt{kiglaze@ncsu.edu} \\
}



\begin{document}

    \maketitle

    % Added sections for project report
    \section{Background \& Introduction}
    The Cornell Lab of Ornithology maintains a large database of bird observations called EBird.
    Birdwatchers submit their bird observations through the Cornell Lab of
    Ornithology mobile app or website, so the data is crowd-sourced. Data points
    exist for thousands of bird species. [7]

    This study is focused on Ospreys, California Condors, and Atlantic Puffins in
    the United States, Mexico, Canada, and Cuba. Ospreys and Atlantic Puffins both
    migrate, but California Condors do not. California Condors are endangered, and Atlantic
    Puffins have a vulnerable population but are not endangered. Ospreys currently
    have a low concern level regarding their population, but their population has previously
    been under threat in the 1950’s. Given this variety of bird characteristics,
    models can be created to predict the presence of bird sightings during a given
    week and location for each of these bird species.

    Limitations of studying the eBird data include that observations depend on
    humans being present and able to access Cornell’s bird app or website. More
    humans birdwatching may be associated with more observations in the dataset.
    The data also depends on people using the bird app to use the app properly.

    Predicting the presence of bird species sightings per week is useful to better understand the
    ecological makeup of a geographical region in relation to time. It also
    provides information about the bird species’ migration patterns in relation to
    a particular geographic region, and the health of the bird population.

    Martin’s paper uses random forests as one of the machine learning models to predict special-temporal
    patters. [1] This helps inform one of the models that this research will use.

    \section{Methods}
    The Osprey .csv
    file geographical observation data was grouped by observation date week and
    then mapped onto maps of North America. Observation countries of focus were
    limited to the United States, Canada, Mexico, and Cuba. These geographical
    distribution maps were stitched together in chronological order to make animations
    showing seasonal migration patterns. This process was repeated for seasons. It
    was also repeated for California Condors and Atlantic Puffins for weeks and
    seasons as well.

    To clean up the
    geographical observation data, all four of the previous animations were
    re-created, but with “noise” points being excluded by applying DBSCAN. The
    distance calculation used for DBSCAN was the “Haversine” distance, which
    accounts for the nature of latitude, longitude, and the roundness of the Earth.
    These post-DBSCAN geographical animations gave a clearer visualization of bird
    migration patterns by species. These cleaned up migration pattern
    visualizations.

    With this done, specific locations were singled out for the next step using the
    visualization maps. The following geographical location and bird species
    combinations were human-selected for further examination: Ospreys at Glacier
    Bay in Alaska, California Condors at the Grand Canyon, and Atlantic Puffins in
    the Massachusetts coastal area. For each of these bird species and geographical
    area combinations, data was transformed to determine whether bird species
    sightings existed for the corresponding geographical area each week between
    March 2015 and August 2025. This was created to help predict the likelihood of
    Osprey sightings each week of the year. To do this, this data was fit to
    logistic regression model.

    Next, GIS data (including temperature and other weather data) will be joined with the
    bird geographical and temporal observation data to provide more variables for
    additional model exploration of each studied species in their respective
    geographical location. These additional variables will help predict the
    presence of bird species sightings per week in a more detailed fashion. Additional
    models utilizing these additional variables will include most of the following:
    Naïve Bayes, Decision Trees, Random Forest, and
    Gradient Boosted Trees, and possibly XGBoost. They will continue to predict the
    likelihood of each of these bird species being spotted in their corresponding
    geographical region of focus.

    \section{Experiment Setup}
    A large 200GB
    dataset of bird sighting data, named Basic Dataset, was downloaded from \href{https://ebird.org/data/download}{https://ebird.org/data/download} after requesting access permission from
    Cornell’s Lab of Ornithology. This data includes latitude, longitude, bird
    species name, observation date, country name, some additional variables
    describing the geographical location, and sometimes the bird’s behavior. Next,
    a Python script was used to extract all data for specific species of birds by
    filtering the large dataset in chunks. This data was exported it to a .csv
    file. The set of bird species included were Ospreys, California Condors, and
    Atlantic Puffins. Then, the Osprey observations, California Condor
    observations, and Atlantic Puffin observations were each output to their own
    single .csv file.

    After bird species were sorted into their own .csv files, additional columns
    were added to describe the observation dates to make for easier grouping by
    dates. Some of these included the month, the year, the season, and the week
    number. Columns for the latitude and longitude in radians were also added to
    help with DBSCAN calculations.

    \section{Results}
    Here are examples of DBSCAN being applied. The first example is from observation days
    being grouped by week, and the second example has days grouped by season.\\
    Week 33, 2024: Osprey Observation Map
    % Side-by-side images for Osprey week 33, 2024
    \begin{samepage}
        \begin{figure}[h]
            \centering
            \begin{minipage}{0.48\textwidth}
                \centering
                \includegraphics[width=\textwidth]{images/no_dbscan_osprey_week33_2024.png}
                \caption{Without DBSCAN}
                \label{fig:no_dbscan_osprey_wk33_24}
            \end{minipage}\hfill
            \begin{minipage}{0.48\textwidth}
                \centering
                \includegraphics[width=\textwidth]{images/dbscan_osprey_week33_2024.png}
                \caption{With DBSCAN}
                \label{fig:dbscan_osprey_wk33_24}
            \end{minipage}
        \end{figure}

        \vspace{-1em}
        Winter 2023: Osprey Observation Map
        % Side-by-side images for Osprey Winter 2023
        \begin{figure}[h]
            \centering
            \begin{minipage}{0.48\textwidth}
                \centering
                \includegraphics[width=\textwidth]{images/no_dbscan_osprey_winter_2023.png}
                \caption{Without DBSCAN}
                \label{fig:no_dbscan_osprey_winter23}
            \end{minipage}\hfill
            \begin{minipage}{0.48\textwidth}
                \centering
                \includegraphics[width=\textwidth]{images/dbsan_osprey_winter_2023.png}
                \caption{With DBSCAN}
                \label{fig:dbscan_osprey_winter23}
            \end{minipage}
        \end{figure}
    \end{samepage}

    Animations linked in YouTube below demonstrate the DBSCAN maps over time from March 2023-August
    2025 for each species. The animations helped inform selecting locations of
    further focus for the next step, cyclic logistic regression.\\

    \begin{samepage}
        Weekly observation date groupings: \\
        Osprey, DBSCAN, Weekly: \url{https://youtu.be/ezr50wL6RH8} \\
        California Condor, DBSCAN, Weekly: \url{https://youtu.be/-5MAU0RcSOI} \\
        Atlantic Puffin, DBSCAN, Weekly: \url{https://youtu.be/eX5zdkZRiiQ} \\
        Seasonal observation date groupings: \\
        Osprey, DBSCAN, Seasonal: \url{https://youtu.be/pz5BUI0dS8o} \\
        California Condor, DBSCAN, Seasonal: \url{https://youtu.be/aYx9zGKV7Q0} \\
        Atlantic Puffin, DBSCAN, Seasonal: \url{https://youtu.be/XfmMUglQXGs} \\
    \end{samepage}

    Cyclical logistic regression was carried out for the presence of Osprey sightings given
    the week number. This plot demonstrates a peak in the likelihood of viewing at
    least one Osprey in Glacier Bay around approximately week 29 or week 30, which is mid to late July.

    \begin{figure}[H]
        \centering
        % First row
        \begin{subfigure}{0.48\textwidth}
            \includegraphics[width=\textwidth]{images/log_reg_osprey_glacier_bay,_alaska.png}
            \caption{Osprey: Glacier Bay, Alaska}
            \label{fig:cyc_log_reg_osprey}
        \end{subfigure}
        \hfill
        \begin{subfigure}{0.48\textwidth}
            \includegraphics[width=\textwidth]{images/log_reg_california_condor_grand_canyon.png}
            \caption{California Condor: Grand Canyon}
            \label{fig:cyc_log_reg_condor}
        \end{subfigure}

        \vspace{0.5em}

        % Second row
        \begin{subfigure}{0.48\textwidth}
            \includegraphics[width=\textwidth]{images/log_reg_atlantic_puffin_massachusetts_coastal_area.png}
            \caption{Atlantic Puffin: Massachusetts Coastal Area}
            \label{fig:cyc_log_reg_puffin}
        \end{subfigure}

        \caption{Cyclical Logistic Regression: Likelihood of Bird Observation Occurrence in Specified Geographical Regions by Week for Three Species/Locations}
        \label{fig:cyc_log_reg_grid}
    \end{figure}
    Results for Naïve Bayes, Decision Trees, Random Forest, and Gradient Boosted Trees are still in
    progress.

    \section{Conclusion}
    Although California Condors have many fewer sightings than Osprey overall, they are
    consistently likely to be spotted in the Grand Canyon at least once a week, any
    given week during the year. Ospreys are much more abundant than California
    Condors, but only likely to be spotted in Glacier Bay, Alaska during the warmer
    months. Similarly, the DBSCAN maps showing bird observations support what is
    shown in the cyclic logistic regression plots in both cases. However, the
    Atlantic Puffin data was slightly surprising because sightings of Atlantic
    Puffins in the Massachusetts coastal area carried on into the winter months.
    Atlantic Puffin binary presence per week seemed to be more common in winter
    months. This aligns partially with how the DBSCAN map animations of Atlantic
    Puffin migrations in the sense that Atlantic Puffins tend to migrate further
    North than Massachusetts in the summer, in Maine and coastal South-Eastern
    Canada.

    \appendix
    \section{Appendix -- Plan Revision}
    DBSCAN and logistic regression were still used, but K-means
    was dropped from the models list, because it requires the number of clusters to
    be pre-determined. Also “cyclic” logistic regression was used instead of
    regular logistic regression, to account for the seasonal nature of bird sighting
    occurrences. Remaining models to implement for prediction still include Naïve Bayes,
    Decision Trees, Random Forest, and Gradient Boosted Trees. However, I may
    choose to replace one with XGBoost. Also, one of these models may be dropped
    from the final results depending on how useful other models are.

    Additionally, the likelihood of the \textit{existence} of a bird sighting given a
    particular \textit{week} is being studied. This is happening in place of studying
    the likelihood of spotting a bird given a particular \textit{day}. The existence
    of a bird sighting in a given week simply means there must be at least one bird
    observation recorded, which differs from examining the number of observations
    in a place for that species.


    \section*{References}
    \medskip
    {
        \small
        [1] Martín, B., J. González-Arias, and J. A. Vicente-Vírseda.
    "Machine learning as a successful approach for predicting complex
    spatio-temporal patterns in animal species abundance." Machine learning 44
        (2021): 289-301.

        [2] Fuentes, Miguel, et al. "BirdFlow: Learning seasonal
    bird movements from eBird data." Methods in Ecology and Evolution 14.3
        (2023): 923-938.

        [3] Ahmed, dis- Nahian, et al. "Spatial clustering of
    citizen science data improves downstream species tribution models."
    Proceedings of the AAAI Conference on Artificial Intelligence. Vol. 39. No. 27.
    2025.

        [4] Cornell Lab. Osprey. All About Birds. \url{https://www.allaboutbirds.org/guide/Osprey/}

    [5] Cornell Lab. California Condor. All About Birds. \url{https://www.allaboutbirds.org/guide/California_Condor/}

    [6] Cornell Lab. Atlantic Puffin. All About Birds. \url{https://www.allaboutbirds.org/guide/Atlantic_Puffin/}

    [7] eBird. About eBird. \url{https://ebird.org/about}
    }
\end{document}
